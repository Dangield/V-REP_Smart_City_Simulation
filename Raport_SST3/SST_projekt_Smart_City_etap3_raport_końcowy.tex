% !TeX encoding = utf8
% !TeX spellcheck = pl_PL

\documentclass[11pt, a4paper, twoside]{article}
\usepackage{graphicx, color, rotating} 
\usepackage[MeX, plmath]{polski} %MeX - tryb pełnej polonizacji
\usepackage[OT4]{fontenc} % T1 - skład fontami EC; OT4 - układ fontów PL
\usepackage{ae,aecompl}
\usepackage[utf8]{inputenc}
\usepackage{lmodern} %czcionka latin modern; jednolita dla tekstu i wzorów;
\usepackage{a4wide}
\usepackage{amsmath}
\usepackage{amssymb, latexsym}
\usepackage{array}
\usepackage{bm}
\usepackage[shortlabels]{enumitem}
\usepackage[font=footnotesize]{caption} %rozmiar czcionki podpisów pod rysunkami
\usepackage{float}
\usepackage{subfig} %obrazki obok siebie
\usepackage[hidelinks]{hyperref}
\usepackage{indentfirst}
\usepackage{geometry}
\geometry{left=25mm, right=25mm, top=25mm, bottom=25mm}

\usepackage[final]{pdfpages}
\usepackage{pdflscape}
\prefixing %notacja prefiksowa w pakiecie 'polski'
\frenchspacing
\linespread{1.1}
\renewcommand{\figurename}{Rys.}
\hyphenation{steer-ing}
%\renewcommand*\thesubsection{\arabic{subsection}} % zmiana numeracji podsekcji 0.X -> X

\begin{document}
	
	\begin{center} 
		{\Large Wydział Elektroniki i Technik Informacyjnych}
		\vskip0.2cm
		{\LARGE \textbf{Sieci i sterowanie systemami (SST): Smart City --~raport końcowy po etapie III projektu } } 
		\vskip0.3cm
		{\Large Szymon Jarocki, Daniel Giełdowski, Maciej Kłos, Michał Okoński}
		\vskip0.8cm
	\end{center} 	
	
Celem ostatniego etapu była ocena skuteczności działania utworzonego oprogramowania. W~związku z~tym przeprowadziliśmy badania eksperymentalne, których wyniki przedstawiamy w~rozdziale~\ref{sec:Wyniki} niniejszego raportu końcowego. Celem uzupełnienia zamieszczamy także podsumowanie projektu i ocenę pracy z~symulatorem.

\section{Omówienie wyników testów}
\label{sec:Wyniki}

\section{Podsumowanie i ocena pracy z symulatorem}
	
	
	
		
\end{document}


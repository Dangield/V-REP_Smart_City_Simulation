% !TeX encoding = utf8
% !TeX spellcheck = pl_PL

\documentclass[11pt, a4paper, twoside]{article}
\usepackage{graphicx, color, rotating} 
\usepackage[MeX, plmath]{polski} %MeX - tryb pełnej polonizacji
\usepackage[OT4]{fontenc} % T1 - skład fontami EC; OT4 - układ fontów PL
\usepackage{ae,aecompl}
\usepackage[utf8]{inputenc}
\usepackage{lmodern} %czcionka latin modern; jednolita dla tekstu i wzorów;
\usepackage{a4wide}
\usepackage{amsmath}
\usepackage{amssymb, latexsym}
\usepackage{array}
\usepackage{bm}
\usepackage[shortlabels]{enumitem}
\usepackage[font=footnotesize]{caption} %rozmiar czcionki podpisów pod rysunkami
\usepackage{float}
\usepackage{subfig} %obrazki obok siebie
\usepackage[hidelinks]{hyperref}
\usepackage{indentfirst}
\usepackage{geometry}
\geometry{left=25mm, right=25mm, top=25mm, bottom=25mm}

\usepackage[final]{pdfpages}
\usepackage{pdflscape}
\prefixing %notacja prefiksowa w pakiecie 'polski'
\frenchspacing
\linespread{1.1}
\renewcommand{\figurename}{Rys.}
\hyphenation{steer-ing}
%\renewcommand*\thesubsection{\arabic{subsection}} % zmiana numeracji podsekcji 0.X -> X

\begin{document}
	
	\begin{center} 
		{\Large Wydział Elektroniki i Technik Informacyjnych}
		\vskip0.2cm
		{\LARGE \textbf{Sieci i sterowanie systemami (SST): Smart City --~raport z etapu II projektu  } } 
		\vskip0.3cm
		{\Large Szymon Jarocki, Daniel Giełdowski, Maciej Kłos, Michał Okoński}
		\vskip0.8cm
	\end{center} 	
	
	Celem tego etapu było przygotowanie koncepcji rozwiązania zadania. W~tym celu stworzyliśmy model dzielnicy miasta oraz model pojazdu. Opracowaliśmy także algorytm sterowania, które zostanie zastosowany przy dalszej realizacji projektu oraz przygotowywaliśmy możliwość wykorzystania tego algorytmu w~symulatorze. Niniejsze prace zostały omówione w~sekcjach od~pierwszej do~czwartej. Z~kolei część piąta jest krótkim komentarzem odnośnie dalszych prac, których pomyślne zrealizowanie będzie równoznaczne z~zakończeniem projektu. 
	\section{Model dzielnicy miasta}
	Z~racji tego, że nie udało nam się znaleźć istniejącego już modelu dzielnicy miasta w~dostępnych źródłach (np. internetowych), stworzyliśmy taki model samodzielnie. Ze~względu na~ograniczenia programu V-REP wielkość modelu jest ograniczona. Plansza ma~wymiary prostokąta i~wobec tego przyjęliśmy, że wzdłuż jej krawędzi w~pewnym od~nich oddaleniu wytyczona jest droga obwodowa. Ponadto w~celu zapobieżenia wyjechaniu pojazdów poza mapę, co~skutkowałoby spadnięciem ich ,,w~przepaść'', na~skraju mapy wzdłuż krawędzi stoi mur. Co~więcej, aby model bardziej przypominał rzeczywistą dzielnicę miasta, dodane zostały budynki i~drzewa. Widok utworzonego modelu przedstawiono na~rysunku~\ref{fig:district}. 
	\begin{figure}[!h]
		\centering
		\includegraphics[width=.95\linewidth]{cityDistrict}
		\caption{Widok modelu dzielnicy miasta w~symulatorze V-REP}
		\label{fig:district}
	\end{figure}
	
	Założyliśmy, że ulice przecinają się pod kątem prostym zgodnie z~szachownicowym planem miasta, przy czym odległości pomiędzy kolejnymi równoległymi ulicami nie są stałe. Ulice traktujemy jako jednojezdniowe i~dwupasmowe --- po jednym pasie ruchu w~każdą stronę. Znaki drogowe i~sygnalizacja świetlna zostały rozmieszczone w~sposób adekwatny do~ruchu prawostronnego. Uwzględniliśmy trzy typu znaków drogowych: ,,stop'', ,,droga z~pierwszeństwem'' i~,,ustąp pierwszeństwa''. Sygnalizatory są z~kolei standardowego typu, tzn. mają trzy kolory świateł: zielone, żółte i~czerwone. Utworzona infrastruktura drogowa pozwoliła na~wyodrębnienie pięciu rodzajów skrzyżowań (ze~względu na~poruszanie się po~nich):
	\begin{itemize}
		\item z~sygnalizacją świetlną -- po~jednym sygnalizatorze na~każdy z~czterech ,,wlotów'',
		\item równorzędne -- bez znaków drogowych i sygnalizatorów; pierwszeństwo przejazdu zgodnie z~,,zasadą prawej ręki'',
		\item z drogą podporządkowaną występującą po obu stronach:
		\begin{itemize}
			\item ze~znakami ,,stop'',
			\item ze~znakami ,,ustąp pierwszeństwa'',
		\end{itemize}
		\item wjazd na~,,obwodnicę'' dzielnicy -- droga obwodowa zawsze z~pierwszeństwem przejazdu, podporządkowana zaś ze~znakiem ,,stop''.
	\end{itemize}

	\newpage
\section{Model pojazdu}
Na potrzeby symulacji utworzony został odpowiedni model samochodu. W tym celu wykorzystane zostały możliwości edycji obiektów dostępne w programie V-REP. Jako baza posłużył jeden z modeli, obecny w oprogramowaniu po jego instalacji, o nazwie 'simple Ackermann steering.ttm'. Model ten wyposażony był w funkcjonalne koła przymocowane do podstawy oraz podstawowe oprogramowanie umożliwiające sterowanie pojazdem z wykorzystaniem klawiatury.

W celu przystosowania modelu do naszych potrzeb wykonany został szereg akcji. Model upodobniono wyglądem do realnego pojazdu poprzez dodanie dodatkowych kształtów, co z pewnością pozytywnie wpłynie na wizualny odbiór symulacji. Dodane zostały między innymi atrapy takich elementów jak szyby, opony oraz reflektory. Z modelu zostały także usunięte zbędne elementy, takie jak linie pokazujące skręcenie osi poszczególnych kół podczas symulacji.

\begin{figure}[!h]
	\centering
	\includegraphics[width=.8\linewidth]{car.jpg}
	\caption{Model samochodu}
	\label{fig:model}
\end{figure}

Samochód biorący udział w symulacji jest z założenia pojazdem autonomicznym, o wysokim stopniu sztucznej inteligencji. Postrzega on swoje środowisko przez posiadane sensory i na podstawie danych identyfikuje wszystkie elementy otoczenia. Stworzenie takiego samochodu wymagałoby jednak ogromnych nakładów czasu i pracy, które są dla nas niedostępne z oczywistych względów. Z tego powodu zdecydowaliśmy się zastosować kilka uproszczeń. Obecny w symulacji model faktycznie posiada jedynie dwa rodzaje czujników: 14 czujników odległości oraz kamerę o niskiej rozdzielczości. Czujniki mają formę wskaźników laserowych, widocznych podczas symulacji, co umożliwia weryfikację poprawności ich działania. Są one rozmieszczone naokoło samochodu w rozkładzie: 2 z przodu, 2 z tyłu, po 3 na każdym boku i jeden na każdym rogu pojazdu. Czujniki zwracają odległość samochodu od innych nieprzenikalnych elementów symulacji, co umożliwia określenie czy przed lub za modelem znajduje się inny pojazd. Kamera zamieszczona na dachu pojazdu zwraca obraz RGB o niskiej rozdzielczości. Jeżeli pozwoli na to czas realizacji projektu, obraz ten wykorzystany zostanie do wykrywania aktualnego koloru sygnalizacji świetlnej. W przeciwnym wypadku stan świateł przekazywany będzie bezpośrednio do wiadomości pojazdu. Inne czujniki, umożliwiające takie rzeczy jak wykrywanie nawierzchni, identyfikację znaków drogowych oraz obserwację ruchu innych modeli znajdujących się w pobliżu, nie zostały fizycznie umieszczone ani zaimplementowane w modelu. Dane te przesyłane będą bezpośrednio z symulacji do poszczególnych pojazdów.

\begin{figure}
	\centering
	\parbox{.33\linewidth}{
		\includegraphics[width=\linewidth]{car_sensors.jpg}
		\caption{Działanie czujników odległości}
		\label{fig:model_odl}
	}
	\parbox{.58\linewidth}{
		\includegraphics[width=\linewidth]{car_kinect.jpg}
		\caption{Obraz widziany z kamery}
		\label{fig:model_kinect}
	}
\end{figure}

	%\newpage
\section{Komunikacja z pakietem MATLAB}

Do komunikacji środowiska symulacyjnego z~pakietem MATLAB wykorzystane zostało API dostarczone wraz z programem V-REP. Główną jego częścią jest klasa pakietu MATLAB o~nazwie \textit{remApi.m}. Realizuje ona podstawowe zagadnienia związane z komunikacją takie jak otwieranie lub zamykanie połączenia z~symulacją, czy też przesyłanie informacji o~obiektach znajdujących się w środowisku symulacyjnym. Na jej podstawie zostały opracowane klasy z~metodami dostosowane do potrzeb tworzonego systemu. Informacje przesyłane ze środowiska symulacyjnego do algorytmu sterującego zostały wymienione poniżej:
\begin{itemize}
	\item plansza:
	\begin{itemize}
		\item rozmiar oraz położenie dróg,
		\item położenie, orientacja oraz rodzaj znaków drogowych,
		\item \textit{(opcjonalnie) informacje o stanie świateł drogowych},
	\end{itemize}
	\item samochody:
	\begin{itemize}
		\item położenie, prędkość oraz orientacja,
		\item orientacja kół,
		\item dane z 14 czujników odległości,
		\item dane z kamery.
	\end{itemize}
\end{itemize}
Na podstawie tych danych algorytm wylicza sygnały sterujące każdego z~samochodów. Dane te są następnie wysyłane i~zastosowane w środowisku symulacyjnym.
	%\newpage
\section{Algorytm sterowania}

Algorytm sterowania pojazdem w sieci miejskiej składa się w ogólności z 3 części:

\begin{itemize}
	\item 
	Wybór \textit{trasy} (route planning) -- polega na wyborze kolejnych dróg, skrzyżowań etc. po~których będzie poruszał się pojazd przemieszczający się z punktu A do B. Jest to równoznaczne z~wyborem kolejnych wierzchołków i krawędzi w grafie. Np. autostrada A2 jest \textit{trasą}, którą można przejechać z Warszawy do Poznania. 
	\item 
	Wybór \textit{trajektorii} (path planning) - polega na wyborze pewnej krzywej, po której pojazd ma się poruszać w ramach \textit{trasy}. Np. jest to \textit{krzywa}, po której porusza się samochód skręcający na skrzyżowaniu.
	\item 
	Wybór \textit{sterowań} pojazdu - polega na zadaniu pojazdowi skrętu kół i prędkości tak żeby w~odpowiedni sposób poruszał się po trajektorii i reagował na zmieniające się otoczenie (np. nagłe hamowanie w momencie wykrycia przeszkody).
\end{itemize}

Kierując się powyższym podziałem zaprojektowano strategię sterowania mającą dwa główne cele: zapewnienie bezpiecznego i bezkolizyjnego przejazdu, a także minimalizację czasów poszczególnych przejazdów. W celu uproszczenia strategi (ale bez straty ogólności) przyjęto następujące założenia:

\begin{itemize}
	\item 
	Sieć dróg reprezentowana jest przez graf skierowany z ważonym krawędziami.	
	\item 
	Pojazd może zmienić kierunek jazdy tylko w momencie dojeżdżania do węzła grafu (skrzyżowania). Dopuszczalne są: zawracanie, skręt i jazda prosto 
	\item 
	Pierwszeństwo na węźle określane jest na podstawie znaków, sygnalizacji, priorytetów dróg etc.
	\item 
	Samochody posiadają pełną wiedzę o sieci dróg oraz występujących na nich ograniczeniach prędkości.
	\item 
	Wszystkie drogi mają co najwyżej jeden pas w każdym kierunku.
	\item 
	Decyzję o zmianie trasy (route) samochody podejmują	w momencie dojazdu do węzła.
	\item
	Pojazdy dysponują pełna wiedzą o stanie (liczbie i prędkościach pojazdów) na drogach wychodzących ze skrzyżowania do którego się zbliżają. Na tej podstawie mogą lokalnie aktualizować wagi krawędzi we własnym grafie dróg. W rzeczywistym świecie możliwe jest bowiem określenie zakorkowania najbliższych dróg "na oko".
	
\end{itemize}

\noindent W kolejnych podsekcjach szczegółowo przedstawiono kolejne części strategii sterowania.

\subsection{Wybór trasy}

Wybór trasy polega na znalezieniu najkrótszej ścieżki w grafie ważonym - stosuje się do tego np. funkcję \textit{shortestpath} w Matlabie. Przetestowane i porównane zostaną 3 strategie doboru wag w grafie. W pierwszym, najprostszym przypadku wagi krawędzi $w_i$ są równe długości dróg $l_i$:

\begin{equation}
w_i = l_i
\end{equation}

W drugim przypadku wagi niosą również informację o maksymalnej dopuszczalnej prędkości na drodze $v_{i,MAX}$ - są równe minimalnemu dopuszczalnemu czasowi przejazdu:

\begin{equation}
w_i = \frac{l_i}{v_{i,MAX}}
\end{equation}

Trzeci przypadek jest rozszerzeniem przypadku drugiego. Dla każdego samochodu, wagi krawędzi wychodzących z węzła do którego pojazd się kieruje, są zaktualizowano o informację na~temat średniej prędkości przejazdu $v_{i,avg}$ samochodów po danej krawędzi - pojazd rozszerza swoją wiedzę o pewną obserwację otoczenia (korków). Wszystkie pozostałe krawędzie przyjmują wagi jak w przypadku drugim:

\begin{equation}
w_i = \frac{l_i}{v_{i,avg}}
\end{equation}

Trzeci przypadek można rozszerzyć o jakąś formę pamięci pojazdu o dotychczas miniętych krawędziach. Decyzja o zmianie trasy podejmowana jest za każdym razem gdy pojazd zbliża się do węzła.

\begin{figure}[!h]
\centering
	\centering
	\includegraphics[width=.8\linewidth]{graf.jpg}
	\caption{Przykładowy graf reprezentujący sieć dróg, źródło: A. Barsi et al. An offline path planning method for autonomous vehicles}
	\label{fig:graf}
\end{figure}

\subsection{Wybór trajektorii}

Trajektorie generowane będą przy pomocy funkcji \textit{pathplannerRRT} i \textit{smoothpathspline} w~Matlabie. Do tych funkcji przesyłane będą informację o aktualnej trasie, mapie dróg, a także informację z czujników wykrywających przeszkody (inne pojazdy). Na wyjściu funkcje generują pewną krzywą.

\begin{figure}[!h]
	\centering
	\includegraphics[width=.8\linewidth]{trajektoria.png}
	\caption{Przykładowa trajektoria wygenerowana w programie Matlab, źródło: Mathworks.com.}
	\label{fig:trajektoria}
\end{figure}

\subsection{Wybór sterowań}

Na podstawie wygenerowanej w poprzednim kroku krzywej, korzystając z funkcji \textit{lateralControllerStanley} przyjmującej trajektorię, wygenerowany zostanie skręt kół. Prędkość pojazdu będzie zadawana pewną logiką biorącą pod uwagę znaki drogowe, sygnalizacje świetlną, a także obecność innych pojazdów - ochrona przed kolizją. Wybrany skręt oraz prędkość wysyłane będą do~symulatora V-REP.
 

	
	\section{Plan dalszych prac}
	Kolejne nasze działania będą związane z~implementacją algorytmu sterowania wraz z~uwzględnieniem symulacji w~stworzonym modelu dzielnicy miasta. Przeprowadzimy odpowiednie testy i~dokonamy oceny poprawności rozwiązania. Weryfikacja powinna pozwolić na~znalezienie ewentualnych błędów lub nieoptymalności w~poruszaniu się pojazdów. Dzięki temu będzie możliwe stworzenie bardziej wartościowego symulatora. Finalnie nasze dalsze prace zostaną udokumentowane w~formie raportu końcowego.
		
\end{document}


% !TeX encoding = utf8
% !TeX spellcheck = pl_PL

\documentclass[11pt, a4paper, twoside]{report}
\usepackage{graphicx, color, rotating} 
\usepackage[MeX, plmath]{polski} %MeX - tryb pełnej polonizacji
\usepackage[OT4]{fontenc} % T1 - skład fontami EC; OT4 - układ fontów PL
\usepackage{ae,aecompl}
\usepackage[utf8]{inputenc}
\usepackage{lmodern} %czcionka latin modern; jednolita dla tekstu i wzorów;
\usepackage{a4wide}
\usepackage{amsmath}
\usepackage{amssymb, latexsym}
\usepackage{array}
\usepackage{bm}
\usepackage[shortlabels]{enumitem}
\usepackage[font=footnotesize]{caption} %rozmiar czcionki podpisów pod rysunkami
\usepackage{float}
\usepackage{subfig} %obrazki obok siebie
\usepackage[hidelinks]{hyperref}
\usepackage{indentfirst}
\usepackage{geometry}
\geometry{left=25mm, right=25mm, top=25mm, bottom=25mm}

\usepackage[final]{pdfpages}
\usepackage{pdflscape}
\prefixing %notacja prefiksowa w pakiecie 'polski'
\frenchspacing
\linespread{1.1}
\renewcommand{\figurename}{Rys.}
\hyphenation{steer-ing}

\makeatletter
\newcommand*{\toccontents}{\@starttoc{toc}}
\makeatother

\begin{document}

	\begin{center} 
		{\Large Wydział Elektroniki i Technik Informacyjnych}
		\vskip0.2cm
		{\LARGE \textbf{Modelowanie i Symulacja Komputerowa (MISK): Smart City --~raport końcowy z~projektu   } } 
		\vskip0.3cm
		{\Large Szymon Jarocki, Daniel Giełdowski, Maciej Kłos, Michał Okoński}
		\vskip3.0cm
	\end{center}	
	\begin{flushleft}
		{\LARGE \textbf{Spis treści} } 
	\end{flushleft}
	
	\toccontents
	

\chapter{Etap pierwszy}	
\section{Analiza zagadnienia}
\label{sec:analiza}
Zadanie polega na~zaprojektowaniu oraz zaimplementowaniu symulatora grupy autonomicznych samochodów (bez kierowcy), które mają poruszać się po ulicach pewnego miasta i~docierać pod zadane adresy. Wybrany został wariant sterowania autonomicznego, w~którym pojazdy same decydują o~swojej trasie, mając do~dyspozycji odpowiednie czujniki umożliwiające unikanie kolizji. Przyjmujemy, że po~drogach poruszają się jedynie rozważane pojazdy. Głównym celem sterowania --- oprócz jazdy bezkolizyjnej --- jest to, aby samochody osiągały swoje punkty docelowe jak najszybciej.

Przy realizacji projektu wykorzystane zostaną środowiska V-REP i~MATLAB. Pierwsze z~nich posłuży do~symulacji poruszania się samochodów, natomiast drugie z~wymienionych będzie wykorzystane do~implementacji algorytmów wyznaczania tras dla pojazdów. Prace, które są do~wykonania można określić w~następującym porządku:
\begin{enumerate}[1)]
	\item implementacja w symulatorze planu dzielnicy miasta oraz modelu pojazdu przy zapewnieniu wystarczającego stopnia złożoności symulowanego ,,świata'',
	\item dobór algorytmów wyznaczania tras dla poruszających się pojazdów w~kontekście posiadanych przez nie czujników i~w~związku z~minimalizacją czasu osiągania punktów docelowych,
	\item implementacja wybranych algorytmów i~konsolidacja oprogramowania,
	\item testowanie poprawności działania symulatora i ocena optymalności rozwiązania.
\end{enumerate}
Co ważne, powyższa kolejność nie jest sekwencją, w~której kolejne etapy realizowane są dokładnie raz w~przedstawionej kolejności. Należy raczej zakładać postępowanie iteracyjne, wiążące się z~powrotem do~wcześniej wykonywanych zadań w~celu dokonania potrzebnych poprawek, które mogą przyczynić się do~uzyskania lepszego rozwiązania.

\section{Przygotowanie środowiska roboczego}
\label{sec:przygotowanie}
Przygotowanie środowiska roboczego polegało przede wszystkim na~instalacji i~zapoznaniu się z~programem V-REP. Z~kolei pakiet MATLAB był już do~naszej dyspozycji z~racji wykonywania z~jego użyciem innych projektów. W~dalszej perspektywie konieczne będzie utrzymanie tych środowisk,  polegające m.in. na~dokonywaniu ewentualnych aktualizacji czy też konfiguracji przydatnych funkcjonalności.

Prace przygotowawcze polegały także na~sprawdzeniu możliwości wspólnego zastosowania obu programów oraz na~przygotowaniu wstępnego planu dzielnicy miasta. Okazało się, że należy utworzyć własny plan z~racji trudności w~odnalezieniu istniejących już rozwiązań (o~ile takowe są), które można by odpowiednio zaadaptować na~potrzeby niniejszego projektu.

\subsection{Instalacja}
\label{subsec:instalacja}
Uruchomienie pracy nad projektem wymagało instalacji dwóch programów: MATLAB oraz \mbox{V-REP}. Ze względu na aktualnie zainstalowane systemy operacyjne instalacja została wykonana na systemie Windows. Licencja programu MATLAB jest dla nas dostępna jako studentów Politechniki Warszawskiej. Przy instalacji tego oprogramowania nie zostały podjęte żadne niestandardowe kroki. Jeżeli chodzi o program V-REP, to z racji braku wykupionej licencji zdecydowaliśmy się na wersję V-REP PRO EDU udostępnioną przez producenta, która pozwala na~wykorzystywanie pełnych możliwości oprogramowania do~celów akademickich. Także w wypadku tej instalacji nie zostały podjęte żadne niestandardowe kroki.

\subsection{Zapoznanie się ze środowiskiem i tworzenie świata}
\label{subsec:scena}
Jak już wcześniej stwierdziliśmy, środowisko MATLAB jest nam dobrze znane ze względu na~fakt wykonywania w nim innych projektów naukowych. Skupiliśmy się zatem na poznaniu oprogramowania V-REP. W tym celu utworzyliśmy w nim własny świat (zwany w środowisku \mbox{V-REP} sceną), na którym rozpoczęliśmy budowanie mapy. Jak wspomnieliśmy, mimo poszukiwań, nie udało nam się dotychczas znaleźć sposobu na udostępnienie programowi planów realnych miast (na przykład poprzez konwersję map dostępnych w internecie), co skłoniło nas do~podjęcia próby zbudowania swojego. Początkowo przetestowaliśmy następujące funkcjonalności programu:
\begin{itemize}
	\item dodawanie przeszkód statycznych (budynków oraz drzew),
	\item wstawianie dróg i dostosowanie ich koloru (w formie płaszczyzn),
	\item dodawanie pojazdów (dostępny w programie V-REP model pojazdu o napędzie Ackermanna),
	\item uruchamianie symulacji,
	\item sterowanie pojazdami (egzekwowane za pomocą klawiatury),
	\item obecność kolizji pojazdów z otoczeniem i ze sobą nawzajem.
\end{itemize}

\subsection{Integracja oprogramowania}
\label{subsec:integracja}
Ważnym aspektem projektu jest połączenie symulacji działającej w programie V-REP ze~sterowaniem egzekwowanym za pomocą algorytmów zaimplementowanych w MATLAB-ie. Jak okazało się, po krótkich poszukiwaniach, twórcy programu V-REP załączyli do swojego kilka przykładów takiego połączenia wraz z plikami umożliwiającymi pisanie własnych skryptów. W~celu ustanowienia połączenia należy wykonać następujące kroki:
\begin{enumerate}[1)]
	\item Otworzyć odpowiednią scenę w programie V-REP (można skorzystać z istniejącego świata $remoteApiCommandServerExample.ttt$).
	\item Otworzyć odpowiedni port w programie V-REP poprzez dodanie odpowiedniego skryptu do sceny. W skrypcie należy także zawrzeć inne funkcje, które miałyby być wywoływane przez MATLAB-a.
	\item Utworzyć w programie MATLAB odpowiedni skrypt umożliwiający komunikację oraz zlecający wykonanie pożądanych działań, lub skorzystać z jednego z już istniejących.
	\item Skopiować następujące pliki z lokalizacji instalacji programu V-REP do folderu zawierającego wykonywany w~MATLAB-ie skrypt (lub dodać ich lokalizację do źródeł MATLAB-a):
	\begin{itemize}
		\item $..\backslash V-REP\_PRO\_EDU\backslash programming\backslash remoteApiBindings\backslash \\matlab\backslash matlab\backslash remApi.m$
		\item $..\backslash V-REP\_PRO\_EDU\backslash programming\backslash remoteApiBindings\backslash \\matlab\backslash matlab\backslash remoteApiProto.m$
		\item $..\backslash V-REP\_PRO\_EDU\backslash programming\backslash remoteApiBindings\backslash \\lib\backslash lib\backslash Windows\backslash 64Bit\backslash remoteApi.dll$
	\end{itemize}
	\item Uruchomić symulację w programie V-REP.
	\item Uruchomić skrypt sterujący MATLAB-a.
\end{enumerate}
Wykonując wspomniane akcje i używając dostępnych skryptów, udało nam się z powodzeniem przetestować działanie komunikacji pomiędzy obydwoma programami --- zarówno w scenie testowej udostępnionej przez deweloperów programu V-REP, jak i w utworzonej przez nas scenie z mapą. Skrypty testowe umożliwiły sprawdzenie następujących działań:
\begin{itemize}
	\item wysyłanie danych z MATLAB-a do V-REP-a i z powrotem,
	\item wywoływanie przez program MATLAB funkcji udostępnionych przez V-REP,
	\item zdalne tworzenie i edycja obiektów na scenie,
	\item wysyłanie kodu wykonywalnego do V-REP-a i jego egzekucja,
	\item sterowanie wykonywaniem kroków symulacji oraz jej wyłączanie.
\end{itemize}

\section{Model dzielnicy miasta}
Z~racji tego, że nie udało nam się znaleźć istniejącego już modelu dzielnicy miasta w~dostępnych źródłach (np. internetowych), stworzyliśmy taki model samodzielnie. Ze~względu na~ograniczenia programu V-REP wielkość modelu jest ograniczona. Plansza ma~wymiary prostokąta i~wobec tego przyjęliśmy, że wzdłuż jej krawędzi w~pewnym od~nich oddaleniu wytyczona jest droga obwodowa. Ponadto w~celu zapobieżenia wyjechaniu pojazdów poza mapę, co~skutkowałoby spadnięciem ich ,,w~przepaść'', na~skraju mapy wzdłuż krawędzi stoi mur. Co~więcej, aby model bardziej przypominał rzeczywistą dzielnicę miasta, dodane zostały budynki i~drzewa. Widok utworzonego modelu przedstawiono na~rysunku~\ref{fig:district}. 
\begin{figure}[!h]
	\centering
	\includegraphics[width=.95\linewidth]{cityDistrict}
	\caption{Widok modelu dzielnicy miasta w~symulatorze V-REP}
	\label{fig:district}
\end{figure}

Założyliśmy, że ulice przecinają się pod kątem prostym zgodnie z~szachownicowym planem miasta, przy czym odległości pomiędzy kolejnymi równoległymi ulicami nie są stałe. Ulice traktujemy jako jednojezdniowe i~dwupasmowe --- po jednym pasie ruchu w~każdą stronę. Znaki drogowe i~sygnalizacja świetlna zostały rozmieszczone w~sposób adekwatny do~ruchu prawostronnego. Uwzględniliśmy trzy typu znaków drogowych: ,,stop'', ,,droga z~pierwszeństwem'' i~,,ustąp pierwszeństwa''. Sygnalizatory są z~kolei standardowego typu, tzn. mają trzy kolory świateł: zielone, żółte i~czerwone. Utworzona infrastruktura drogowa pozwoliła na~wyodrębnienie pięciu rodzajów skrzyżowań (ze~względu na~poruszanie się po~nich):
\begin{itemize}
	\item z~sygnalizacją świetlną -- po~jednym sygnalizatorze na~każdy z~czterech ,,wlotów'',
	\item równorzędne -- bez znaków drogowych i sygnalizatorów; pierwszeństwo przejazdu zgodnie z~,,zasadą prawej ręki'',
	\item z drogą podporządkowaną występującą po obu stronach:
	\begin{itemize}
		\item ze~znakami ,,stop'',
		\item ze~znakami ,,ustąp pierwszeństwa'',
	\end{itemize}
	\item wjazd na~,,obwodnicę'' dzielnicy -- droga obwodowa zawsze z~pierwszeństwem przejazdu, podporządkowana zaś ze~znakiem ,,stop''.
\end{itemize}

\newpage
\section{Model pojazdu}
Na potrzeby symulacji utworzony został odpowiedni model samochodu. W tym celu wykorzystane zostały możliwości edycji obiektów dostępne w programie V-REP. Jako baza posłużył jeden z modeli, obecny w oprogramowaniu po jego instalacji, o nazwie 'simple Ackermann steering.ttm'. Model ten wyposażony był w funkcjonalne koła przymocowane do podstawy oraz podstawowe oprogramowanie umożliwiające sterowanie pojazdem z wykorzystaniem klawiatury.

W celu przystosowania modelu do naszych potrzeb wykonany został szereg akcji. Model upodobniono wyglądem do realnego pojazdu poprzez dodanie dodatkowych kształtów, co z pewnością pozytywnie wpłynie na wizualny odbiór symulacji. Dodane zostały między innymi atrapy takich elementów jak szyby, opony oraz reflektory. Z modelu zostały także usunięte zbędne elementy, takie jak linie pokazujące skręcenie osi poszczególnych kół podczas symulacji.

\begin{figure}[!h]
	\centering
	\includegraphics[width=.8\linewidth]{car.jpg}
	\caption{Model samochodu}
	\label{fig:model}
\end{figure}

Samochód biorący udział w symulacji jest z założenia pojazdem autonomicznym, o wysokim stopniu sztucznej inteligencji. Postrzega on swoje środowisko przez posiadane sensory i na podstawie danych identyfikuje wszystkie elementy otoczenia. Stworzenie takiego samochodu wymagałoby jednak ogromnych nakładów czasu i pracy, które są dla nas niedostępne z oczywistych względów. Z tego powodu zdecydowaliśmy się zastosować kilka uproszczeń. Obecny w symulacji model faktycznie posiada jedynie dwa rodzaje czujników: 14 czujników odległości oraz kamerę o niskiej rozdzielczości. Czujniki mają formę wskaźników laserowych, widocznych podczas symulacji, co umożliwia weryfikację poprawności ich działania. Są one rozmieszczone naokoło samochodu w rozkładzie: 2 z przodu, 2 z tyłu, po 3 na każdym boku i jeden na każdym rogu pojazdu. Czujniki zwracają odległość samochodu od innych nieprzenikalnych elementów symulacji, co umożliwia określenie czy przed lub za modelem znajduje się inny pojazd. Kamera zamieszczona na dachu pojazdu zwraca obraz RGB o niskiej rozdzielczości. Jeżeli pozwoli na to czas realizacji projektu, obraz ten wykorzystany zostanie do wykrywania aktualnego koloru sygnalizacji świetlnej. W przeciwnym wypadku stan świateł przekazywany będzie bezpośrednio do wiadomości pojazdu. Inne czujniki, umożliwiające takie rzeczy jak wykrywanie nawierzchni, identyfikację znaków drogowych oraz obserwację ruchu innych modeli znajdujących się w pobliżu, nie zostały fizycznie umieszczone ani zaimplementowane w modelu. Dane te przesyłane będą bezpośrednio z symulacji do poszczególnych pojazdów.

\begin{figure}
	\centering
	\parbox{.33\linewidth}{
		\includegraphics[width=\linewidth]{car_sensors.jpg}
		\caption{Działanie czujników odległości}
		\label{fig:model_odl}
	}
	\parbox{.58\linewidth}{
		\includegraphics[width=\linewidth]{car_kinect.jpg}
		\caption{Obraz widziany z kamery}
		\label{fig:model_kinect}
	}
\end{figure}

%\newpage
\section{Komunikacja z pakietem MATLAB}

Do komunikacji środowiska symulacyjnego z~pakietem MATLAB wykorzystane zostało API dostarczone wraz z programem V-REP. Główną jego częścią jest klasa pakietu MATLAB o~nazwie \textit{remApi.m}. Realizuje ona podstawowe zagadnienia związane z komunikacją takie jak otwieranie lub zamykanie połączenia z~symulacją, czy też przesyłanie informacji o~obiektach znajdujących się w środowisku symulacyjnym. Na jej podstawie zostały opracowane klasy z~metodami dostosowane do potrzeb tworzonego systemu. Informacje przesyłane ze środowiska symulacyjnego do algorytmu sterującego zostały wymienione poniżej:
\begin{itemize}
	\item plansza:
	\begin{itemize}
		\item rozmiar oraz położenie dróg,
		\item położenie, orientacja oraz rodzaj znaków drogowych,
		\item \textit{(opcjonalnie) informacje o stanie świateł drogowych},
	\end{itemize}
	\item samochody:
	\begin{itemize}
		\item położenie, prędkość oraz orientacja,
		\item orientacja kół,
		\item dane z 14 czujników odległości,
		\item dane z kamery.
	\end{itemize}
\end{itemize}
Na podstawie tych danych algorytm wylicza sygnały sterujące każdego z~samochodów. Dane te są następnie wysyłane i~zastosowane w środowisku symulacyjnym.
%\newpage
\section{Algorytm sterowania}

Algorytm sterowania pojazdem w sieci miejskiej składa się w ogólności z 3 części:

\begin{itemize}
	\item 
	Wybór \textit{trasy} (route planning) -- polega na wyborze kolejnych dróg, skrzyżowań etc. po~których będzie poruszał się pojazd przemieszczający się z punktu A do B. Jest to równoznaczne z~wyborem kolejnych wierzchołków i krawędzi w grafie. Np. autostrada A2 jest \textit{trasą}, którą można przejechać z Warszawy do Poznania. 
	\item 
	Wybór \textit{trajektorii} (path planning) - polega na wyborze pewnej krzywej, po której pojazd ma się poruszać w ramach \textit{trasy}. Np. jest to \textit{krzywa}, po której porusza się samochód skręcający na skrzyżowaniu.
	\item 
	Wybór \textit{sterowań} pojazdu - polega na zadaniu pojazdowi skrętu kół i prędkości tak żeby w~odpowiedni sposób poruszał się po trajektorii i reagował na zmieniające się otoczenie (np. nagłe hamowanie w momencie wykrycia przeszkody).
\end{itemize}

Kierując się powyższym podziałem zaprojektowano strategię sterowania mającą dwa główne cele: zapewnienie bezpiecznego i bezkolizyjnego przejazdu, a także minimalizację czasów poszczególnych przejazdów. W celu uproszczenia strategi (ale bez straty ogólności) przyjęto następujące założenia:

\begin{itemize}
	\item 
	Sieć dróg reprezentowana jest przez graf skierowany z ważonym krawędziami.	
	\item 
	Pojazd może zmienić kierunek jazdy tylko w momencie dojeżdżania do węzła grafu (skrzyżowania). Dopuszczalne są: zawracanie, skręt i jazda prosto 
	\item 
	Pierwszeństwo na węźle określane jest na podstawie znaków, sygnalizacji, priorytetów dróg etc.
	\item 
	Samochody posiadają pełną wiedzę o sieci dróg oraz występujących na nich ograniczeniach prędkości.
	\item 
	Wszystkie drogi mają co najwyżej jeden pas w każdym kierunku.
	\item 
	Decyzję o zmianie trasy (route) samochody podejmują	w momencie dojazdu do węzła.
	\item
	Pojazdy dysponują pełna wiedzą o stanie (liczbie i prędkościach pojazdów) na drogach wychodzących ze skrzyżowania do którego się zbliżają. Na tej podstawie mogą lokalnie aktualizować wagi krawędzi we własnym grafie dróg. W rzeczywistym świecie możliwe jest bowiem określenie zakorkowania najbliższych dróg "na oko".
	
\end{itemize}

\noindent W kolejnych podsekcjach szczegółowo przedstawiono kolejne części strategii sterowania.

\subsection{Wybór trasy}

Wybór trasy polega na znalezieniu najkrótszej ścieżki w grafie ważonym - stosuje się do tego np. funkcję \textit{shortestpath} w Matlabie. Przetestowane i porównane zostaną 3 strategie doboru wag w grafie. W pierwszym, najprostszym przypadku wagi krawędzi $w_i$ są równe długości dróg $l_i$:

\begin{equation}
w_i = l_i
\end{equation}

W drugim przypadku wagi niosą również informację o maksymalnej dopuszczalnej prędkości na drodze $v_{i,MAX}$ - są równe minimalnemu dopuszczalnemu czasowi przejazdu:

\begin{equation}
w_i = \frac{l_i}{v_{i,MAX}}
\end{equation}

Trzeci przypadek jest rozszerzeniem przypadku drugiego. Dla każdego samochodu, wagi krawędzi wychodzących z węzła do którego pojazd się kieruje, są zaktualizowano o informację na~temat średniej prędkości przejazdu $v_{i,avg}$ samochodów po danej krawędzi - pojazd rozszerza swoją wiedzę o pewną obserwację otoczenia (korków). Wszystkie pozostałe krawędzie przyjmują wagi jak w przypadku drugim:

\begin{equation}
w_i = \frac{l_i}{v_{i,avg}}
\end{equation}

Trzeci przypadek można rozszerzyć o jakąś formę pamięci pojazdu o dotychczas miniętych krawędziach. Decyzja o zmianie trasy podejmowana jest za każdym razem gdy pojazd zbliża się do węzła.

\begin{figure}[!h]
\centering
	\centering
	\includegraphics[width=.8\linewidth]{graf.jpg}
	\caption{Przykładowy graf reprezentujący sieć dróg, źródło: A. Barsi et al. An offline path planning method for autonomous vehicles}
	\label{fig:graf}
\end{figure}

\subsection{Wybór trajektorii}

Trajektorie generowane będą przy pomocy funkcji \textit{pathplannerRRT} i \textit{smoothpathspline} w~Matlabie. Do tych funkcji przesyłane będą informację o aktualnej trasie, mapie dróg, a także informację z czujników wykrywających przeszkody (inne pojazdy). Na wyjściu funkcje generują pewną krzywą.

\begin{figure}[!h]
	\centering
	\includegraphics[width=.8\linewidth]{trajektoria.png}
	\caption{Przykładowa trajektoria wygenerowana w programie Matlab, źródło: Mathworks.com.}
	\label{fig:trajektoria}
\end{figure}

\subsection{Wybór sterowań}

Na podstawie wygenerowanej w poprzednim kroku krzywej, korzystając z funkcji \textit{lateralControllerStanley} przyjmującej trajektorię, wygenerowany zostanie skręt kół. Prędkość pojazdu będzie zadawana pewną logiką biorącą pod uwagę znaki drogowe, sygnalizacje świetlną, a także obecność innych pojazdów - ochrona przed kolizją. Wybrany skręt oraz prędkość wysyłane będą do~symulatora V-REP.
 



\chapter{Etap drugi} 
\section{Omówienie wyników testów} \vspace{-0.2cm}
Celem ostatniego etapu była ocena skuteczności działania utworzonego oprogramowania. Zaprojektowane algorytmy sterowania przetestowaliśmy na 3 przykładowych scenariuszach ruchu uszeregowanych pod kątem stopnia trudności. \vspace{-0.2cm}

\input{wyniki_symulacji2.tex}

\section{Podsumowanie, ocena pracy z symulatorem i perspektywy dalszych prac}
Zadania i cele projektu zostały zrealizowane, gdyż jako efekt końcowy zaimplementowano symulator grupy autonomicznych samochodów, które poruszają się w~obrębie stworzonego modelu dzielnicy miasta. Pojazdy zostały specjalnie zaprojektowane na~potrzeby projektu i~wyposażone w~odpowiednie czujniki. Strategia sterowania jest realizowana w~programie MATLAB, docelowy symulator w~systemie \mbox{V-REP}, a~wszystko to możliwe jest dzięki odpowiedniej komunikacji pomiędzy tymi programami. Do~zalet symulatora należy przede wszystkim jego intuicyjna wizualizacja, która sprawia, że symulacje mogą zostać zaprezentowane szerokiemu gronu odbiorców, a~ich ogląd nie wymaga specjalistycznej wiedzy. Ponadto zastosowane zostało sterowanie autonomiczne oraz w~uproszczeniu zachowano zasady ruchu drogowego według polskiego prawa.
	
Pracę z~symulatorem (program V-REP) możemy uznać za~dość sprawną ze~względu na~aspekty obliczeń fizycznych. Odwzorowane są one dobrze i~jednocześnie czas trwania symulacji był w~naszych warunkach (prosty model pojazdu) zbliżony do~rzeczywistego, co~jest niewątpliwą zaletą w~porównaniu z~innymi znanymi nam symulatorami (np. Gazebo). Z~drugiej strony przygotowanie modelu dzielnicy jest żmudne, a~ponadto brakuje takiej funkcjonalności, jak edycja kształtu obiektu. Co~więcej, nie ma niektórych podstawowych kształtów (np. trójkąta), nie mówiąc już o~pewnych bardziej złożonych funkcjach. Niemniej jednak istnieje sporo gotowych obiektów, które można zaadaptować na~potrzeby tworzonych modeli. 

Stworzony symulator możemy potraktować jako produkt w~wersji bazowej, ponieważ zasadniczo algorytm sterowania działa, jednak możliwe jest dodanie nowych funkcjonalności czy też ulepszenie istniejących. W~szczególności warto byłoby rozważyć więcej rodzajów skrzyżowań i~lepiej odwzorować rzeczywiste sytuacje na~drogach. Ponadto sam pojazd może być rozwijany, chociażby poprzez dodanie różnego rodzaju czujników oraz wierniejsze (bliższe rzeczywistości) eksploatowanie istniejących. Wreszcie można by zastosować bardziej zaawansowane strategie sterowania. Na~przykład zamiast najprostszego wariantu z~wagami krawędzi w~grafie równymi długości dróg, warto zastosować choćby wariant z~wartościami wag odpowiadającymi minimalnemu czasowi przejazdu przy uwzględnieniu obserwacji otoczenia. 
 
	
		
\end{document}


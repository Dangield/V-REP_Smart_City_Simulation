%\newpage
\section{Komunikacja z pakietem MATLAB}

Do komunikacji środowiska symulacyjnego z~pakietem MATLAB wykorzystane zostało API dostarczone wraz z programem V-REP. Główną jego częścią jest klasa pakietu MATLAB o~nazwie \textit{remApi.m}. Realizuje ona podstawowe zagadnienia związane z komunikacją takie jak otwieranie lub zamykanie połączenia z~symulacją, czy też przesyłanie informacji o~obiektach znajdujących się w środowisku symulacyjnym. Na jej podstawie zostały opracowane klasy z~metodami dostosowane do potrzeb tworzonego systemu. Informacje przesyłane ze środowiska symulacyjnego do algorytmu sterującego zostały wymienione poniżej:
\begin{itemize}
	\item plansza:
	\begin{itemize}
		\item rozmiar oraz położenie dróg,
		\item położenie, orientacja oraz rodzaj znaków drogowych,
		\item \textit{(opcjonalnie) informacje o stanie świateł drogowych},
	\end{itemize}
	\item samochody:
	\begin{itemize}
		\item położenie, prędkość oraz orientacja,
		\item orientacja kół,
		\item dane z 14 czujników odległości,
		\item dane z kamery.
	\end{itemize}
\end{itemize}
Na podstawie tych danych algorytm wylicza sygnały sterujące każdego z~samochodów. Dane te są następnie wysyłane i~zastosowane w środowisku symulacyjnym.